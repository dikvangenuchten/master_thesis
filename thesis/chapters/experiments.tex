\chapter{Experimental Setup}\label{chapter:second_real_chapter}

\section{Implementation details}
\subsection{Dataset}
\paragraph{CoCo~\cite{lin2015microsoftcococommonobjects}} is a publicly available dataset which has a multitude of label categories. The semantic masks are extracted from the 2017 Panoptic annotations. This set contains more than 100k images that are densely annotated with both the semantic class and instance. There are a total of 133 different classes which belong to 27 `supercategories'. The four categories ``food'' and ``food-stuff'' and ``furniture'' and ``furniture-stuff'' were merged into ``food'' and ``furniture'' respectively, bringing the number of supercategories to 25. To speed up convergences, we will use these 25 classes. The distribution of the class labels can be seen in Figure~\ref{fig:coco-class-distribution}. Some samples of the dataset can be seen in Figure~\ref{fig:coco-samples}. More samples can be found in Appendix~\ref{appendix:coco_samples}. Furthermore, during training some basic image augmentations were added; random flips, random grayscale, and Gaussian blur.


\begin{figure}[h]
    \centering
    \includegraphics[width=0.9\textwidth]{figures/datasets/coco/class_distribution.png}
    \caption{Class Distribution of CoCo}
    \label{fig:coco-class-distribution}
\end{figure}

\begin{figure}
    \centering
    \subfloat[Training sample 0]{%
        \includegraphics[width=0.5\textwidth]{figures/datasets/coco/samples/train/0.png}%
    }
    \subfloat[Training sample 1]{%
        \includegraphics[width=0.5\textwidth]{figures/datasets/coco/samples/train/1.png}%
    }\\
    \subfloat[Validation sample 0]{%
        \includegraphics[width=0.5\textwidth]{figures/datasets/coco/samples/val/0.png}%
    }
    \subfloat[Validation sample 1]{%
        \includegraphics[width=0.5\textwidth]{figures/datasets/coco/samples/val/1.png}%
    }
    \caption{\label{fig:coco-samples}Ground truths for the dataset samples}
\end{figure}


\subsection{Training Settings}
All models are implemented in PyTorch~\cite{Ansel_PyTorch_2_Faster_2024} using an adapted version of the segmentation models repository~\cite{Iakubovskii:2019}. All code and configurations required to train the models and reproduce the experiments can be found on GitHub\footnote[1]{\url{https://github.com/Generative-AI-TUe/msc-project-1297333}}. The models are trained on an NVIDIA GeForce RTX 2080 TI. The gradients are clipped using the norm, with a maximum value of 10. The AdaMax~\cite{kingma2017adammethodstochasticoptimization} is used with a cosine annealing learning rate, varying between $1e^{-3}$ and $1e^{-4}$. A batch size of 96 is used as it was the maximum that reliably fitted on the GPU. Each model was trained on 10,000 minibatches randomly sampled from the dataset.

\section{Comparison to Baseline Architectures}
For the experiments we make use of three methods: VAES (ours), FPN, and U-Net. Each model was trained eight times with different configurations. We used different settings for the pre-trained weights:
\begin{itemize}
    \item Without (`None') pre-trained weights.
    \item With pre-trained weights retrieved from the classification of ImageNet.
    \item With pre-trained weights retrieved from a $\beta$-VAE, with $\beta=1$ and $\beta=100$.
\end{itemize}
Additionally, the model is trained with `frozen' encoder weights (the encoder weights are not updated), and with `unfrozen' encoder weights (the encoder weights are updated). The resulting Evaluation Jaccard Index can be seen in Table~\ref{tab:baseline_results}. Some example predictions are shown in Figure~\ref{fig:baseline-sample-results-0}.

\begin{table}[ht]
\caption{Eval Jaccard Index for our model and the baselines for various parameters. Higher is better.}
\label{tab:baseline_results}
\begin{tabular}{llrrrr}
\toprule
 & weights & None & imagenet & vae-b1 & vae-b100 \\
frozen & architecture &  &  &  &  \\
\midrule
\multirow[c]{3}{*}{False} & VAES & 0.21 & 0.47 & 0.20 & 0.23 \\
 & fpn & \textbf{0.30} & \textbf{0.52} & \textbf{0.26} & \textbf{0.28} \\
 & unet & 0.23 & 0.49 & 0.21 & 0.24 \\
\multirow[c]{3}{*}{True} & VAES & n.a. & 0.40 & 0.20 & 0.18 \\
 & fpn & n.a. & 0.45 & 0.21 & 0.21 \\
 & unet & n.a. & 0.42 & 0.20 & 0.19 \\
\bottomrule
\end{tabular}
\end{table}

\begin{figure}[h]
    \foreach \i in {0,1,...,4} {
            \centering
            \subfloat[Image]{\includegraphics[width=0.2\textwidth]{figures/baselines/samples/VAES-imagenet-False/\i.png}}
            \subfloat[Ground Truth]{\includegraphics[width=0.2\textwidth]{figures/baselines/samples/VAES-imagenet-False/gt_\i.png}}
            \subfloat[VAES-imagenet]{\includegraphics[width=0.2\textwidth]{figures/baselines/samples/VAES-imagenet-False/pr_\i.png}}
            \subfloat[unet-imagenet]{\includegraphics[width=0.2\textwidth]{figures/baselines/samples/unet-imagenet-False/pr_\i.png}}
            \subfloat[fpn-imagenet]{\includegraphics[width=0.2\textwidth]{figures/baselines/samples/fpn-imagenet-False/pr_\i.png}}
            \\
        }
    \caption{Samples of the validation dataset, with the ground truth and the prediction by the model. The encoder is not frozen.}\label{fig:baseline-sample-results-0}
\end{figure}


First, an Analysis of Variance (ANOVA) is done, of which the results can be seen in Table~\ref{tab:comparison_baselines_anova_all}. From the ANOVA table it can be determined that the pre-training weights and the architecture have a significant effect on the Jaccard Index. Freezing the weights does not have a significant effect. However, this can be due to the limited sample size as the $p$-value is close to $0.05$. Furthermore, there is no significant interaction effect between the parameters. 
\begin{table}[ht]
\caption{Anova results estimating the influence of each parameter.\\Where: \\\hphantom{tabb}`weights' are the pretrained weights (or lack thereof) used.\\\hphantom{tabb}`architecture' is the model architecture used.\\\hphantom{tabb}'frozen' indicates whether the encoder was frozen\\\hphantom{tabb}$A$:$B$ is the interaction effect between $A$ and $B$}
\label{tab:comparison_baselines_anova_all}
\begin{tabular}{lrrrrr}
\toprule
 & df & sum\_sq & mean\_sq & F & PR(>F) \\
\midrule
frozen & 1.00 & 0.00 & 0.00 & 5.29 & 0.08 \\
weights & 3.00 & 0.25 & 0.08 & 175.48 & \textbf{0.00} \\
architecture & 2.00 & 0.01 & 0.01 & 12.69 & \textbf{0.02} \\
frozen:weights & 3.00 & 0.00 & 0.00 & 1.53 & 0.34 \\
frozen:architecture & 2.00 & 0.00 & 0.00 & 2.16 & 0.23 \\
weights:architecture & 6.00 & 0.00 & 0.00 & 0.64 & 0.70 \\
Residual & 4.00 & 0.00 & 0.00 & n.a. & n.a. \\
\bottomrule
\end{tabular}
\end{table}


\begin{table}[ht]
\caption{Coefficients of the OLS showing the influence of the hyperparameters on the Evaluation Jaccard Index.\\Where:\\\hphantom{tabb}Coef. the effectsize.\\\hphantom{tabb}P> |t| the p-value. Bolded if significant ($\alpha\le0.05$).}
\label{tab:comparison_baselines_ols_effects}
\begin{tabular}{lrrrrrr}
\toprule
 & Coef. & Std.Err. & t & P>|t| & [0.025 & 0.975] \\
\midrule
Intercept & 0.22 & 0.02 & 10.54 & \textbf{0.00} & 0.17 & 0.26 \\
weights[T.imagenet] & 0.21 & 0.02 & 9.59 & \textbf{0.00} & 0.17 & 0.26 \\
weights[T.vae-b1] & -0.03 & 0.02 & -1.33 & 0.20 & -0.08 & 0.02 \\
weights[T.vae-b100] & -0.03 & 0.02 & -1.54 & 0.14 & -0.08 & 0.01 \\
architecture[T.fpn] & 0.06 & 0.02 & 3.49 & \textbf{0.00} & 0.02 & 0.09 \\
architecture[T.unet] & 0.02 & 0.02 & 1.48 & 0.16 & -0.01 & 0.06 \\
\bottomrule
\end{tabular}
\end{table}

Secondly, an Ordinary Least Squares Linear Model (OLS) is fitted to determine the effect size of each significant parameter. The results of this are shown in Table~\ref{tab:comparison_baselines_ols_effects}. The intercept is the baseline model. Here it is the model with the VAES architecture and no pre-trained weights. Compared to this baseline, there is no significant effect on the Jaccard Index when using the pre-trained VAE weights instead. Using the `ImageNet' weights does double the Jaccard Index. The choice in architecture matters less. Only the FPN architecture is a significantly better choice. 



\section{Reduction of Data Required}
To determine if pre-training results in a reduction of labelled training data required, a full factorial design is done over the parameters: architecture type, pre-trained weights, and the percentage of the (labelled) dataset used. The resulting Evaluation Jaccard Index of each model configuration can be seen in Table~\ref{tab:data_fraction_results} and is plotted in Figure~\ref{fig:dataset-fraction-results}.
\begin{table}[ht]
\centering
\caption{The evaluation Jaccard Index for the various models and dataset fractions. Higher is better.}
\label{tab:data_fraction_results}
\begin{tabular}{llrrrr}
\toprule
     & fraction &  1.000 &  0.100 &  0.010 &  0.001 \\
architecture & weights &        &        &        &        \\
\midrule
VAES & None &  0.146 &  0.165 &  0.180 &  0.120 \\
     & imagenet &  0.340 &  0.373 &  0.386 &  0.206 \\
     & vae-b10 &  0.137 &  0.160 &  0.180 &  0.118 \\
fpn & None &  0.218 &  0.117 &  0.213 &  0.167 \\
     & imagenet &  0.455 &  0.481 &  0.483 &  0.337 \\
     & vae-b10 &  0.186 &  0.212 &  0.231 &  0.163 \\
unet & None &  0.211 &  0.095 &  0.184 &  0.139 \\
     & imagenet &  0.463 &  0.441 &  0.433 &  0.298 \\
     & vae-b10 &  0.202 &  0.178 &  0.212 &  0.136 \\
\bottomrule
\end{tabular}
\end{table}


\begin{figure}[h]
    \centering
    \includegraphics[width=0.9\textwidth]{figures/data_percentage/line-plot.png}
    \caption{Jaccard Index for reduced training dataset size.}
    \label{fig:dataset-fraction-results}
\end{figure}

An Analysis of Variance (ANOVA) is done, of which the results can be seen in Table~\ref{tab:data_fraction_parameter_significance}. From the ANOVA table it can be determined that the pre-training weights, the architecture, and the size of the dataset all have an effect on the Jaccard Index. However, there is no significant interaction effect between the parameters. Based on these results it cannot be concluded that using pre-trained weights reduces the required amount of training data significantly. 
\begin{table}
    \centering
    \caption{ANOVA results estimating the influence of each parameter.\\Where: \\\hphantom{tabb}`weights' are the pretrained weights (or lack thereof) used.\\\hphantom{tabb}`architecture' is the model architecture used.\\\hphantom{tabb}`$\log_{10}(\text{fraction})$' is the fraction of data used in log scale.\\\hphantom{tabb}$A$:$B$ is the interaction effect between $A$ and $B$}
    \label{tab:data_fraction_parameter_significance}
    \begin{tabular}{lrrrrr}
        \toprule
                                                          & df    & sum\_sq & mean\_sq & F     & PR(>F)        \\
        \midrule
        weights                                           & 2.00  & 0.39    & 0.20     & 75.49 & \textbf{0.00} \\
        architecture                                      & 2.00  & 0.02    & 0.01     & 4.61  & \textbf{0.02} \\
        weights:architecture                              & 4.00  & 0.01    & 0.00     & 0.93  & 0.47          \\
        $\log_{10}(\text{fraction})$                      & 1.00  & 0.02    & 0.02     & 6.45  & \textbf{0.02} \\
        $\log_{10}(\text{fraction})$:weights              & 2.00  & 0.01    & 0.01     & 2.11  & 0.15          \\
        $\log_{10}(\text{fraction})$:architecture         & 2.00  & 0.00    & 0.00     & 0.22  & 0.80          \\
        $\log_{10}(\text{fraction})$:weights:architecture & 4.00  & 0.00    & 0.00     & 0.01  & 1.00          \\
        Residual                                          & 18.00 & 0.05    & 0.00     & n.a.  & n.a.          \\
        \bottomrule
    \end{tabular}
\end{table}

Following this, an OLS is fitted to determine the effect size of each significant parameter. The results of this are shown in Table~\ref{tab:data_fraction_parameter_influence}. This shows that the usage of `ImageNet' weights has the strongest positive effect on the resulting Jaccard Index. Using the pre-trained weights from the VAE does not significantly differ from using random weights. Using a different architecture than the VAES, does increase the Jaccard Index performance of the model by $\sim0.05$. The smallest effect is that of the reduction in data.

\begin{table}[ht]
\centering
\caption{Coefficients of the OLS.\\Where:\\\hphantom{tabb}Coef. the effectsize.\\\hphantom{tabb}P> |t| the p-value. Bolded if significant ($\alpha\leq0.05$).}
\label{tab:data_fraction_parameter_influence}
\begin{tabular}{lrrrrrr}
\toprule
 & Coef. & Std.Err. & t & P>|t| & [0.025 & 0.975] \\
\midrule
Intercept & 0.16 & 0.02 & 7.56 & \textbf{0.00} & 0.12 & 0.20 \\
weights[T.imagenet] & 0.23 & 0.02 & 11.66 & \textbf{0.00} & 0.19 & 0.27 \\
weights[T.vae-b10] & 0.01 & 0.02 & 0.67 & 0.50 & -0.03 & 0.05 \\
architecture[T.fpn] & 0.06 & 0.02 & 3.19 & \textbf{0.00} & 0.02 & 0.10 \\
architecture[T.unet] & 0.04 & 0.02 & 2.05 & \textbf{0.05} & 0.00 & 0.08 \\
$\log_{10}(\text{fraction})$ & 0.02 & 0.01 & 2.71 & \textbf{0.01} & 0.00 & 0.03 \\
\bottomrule
\end{tabular}
\end{table}


\section{Model Characteristics}

As the computational resources in mobile robots tend to be low, the inference speed and memory usage were measured. Inference was done on an NVIDIA GTX 1070, and measured using PyTorch benchmarking tools. The results can be seen in Table~\ref{tab:model_characteristics}. The added variational computations that are required for the VAES architecture, more than doubles the inference speed. The FPN architecture is the fastest by a small margin, but does require slightly more 

\begin{table}[ht]
\centering
\caption{Characteristics of the various architectures with ResNet50 as backbone. Inference measurements were done on a NVIDIA GTX 1070 with an image shape of 128x128.}
\label{tab:model_characteristics}
\begin{tabular}{llrrr}
\toprule
 & Batch Size & vaes & fpn & unet \\
\midrule
Parameters (x$1e^6$) &   & 132.88 & \textbf{26.12} & 32.52 \\
Total MAC (x$1e^9$) &   & 7.85 & \textbf{1.95} & 2.72 \\
\multirow[c]{4}{*}{Inference Speed (ms)} & 1 & 19.22 & 8.04 & \textbf{7.36} \\
 & 2 & 20.94 & \textbf{7.68} & 8.77 \\
 & 8 & 42.30 & \textbf{15.45} & 16.73 \\
 & 32 & 96.92 & \textbf{47.05} & 49.70 \\
\multirow[c]{4}{*}{Memory Usage (mb)} & 1 & 1375.44 & \textbf{670.71} & 856.52 \\
 & 2 & 1381.01 & \textbf{675.03} & 860.71 \\
 & 8 & 1400.93 & \textbf{703.21} & 885.09 \\
 & 32 & 1881.55 & \textbf{973.62} & 1077.38 \\
\bottomrule
\end{tabular}
\end{table}



\section{Ablation Study}

\subsection{Skip Connection}
To understand the importance of the type and number of skip connections that are added after the pre-training, we will show the effect of this by incrementally removing more skip connections. We vary the number of skip connections from 1 to 5 and try all 3 skip connection types. The results can be viewed in Table~\ref{tab:skip_results}. Lowering the number of skip connections has as benefit that it reduces the amount of parameters and computations that are required to run the model.
\begin{table}[ht]
\centering
\caption{Evaluation Jaccard Index for the VAES using different amounts and types of skip connections.}
\label{tab:skip_results}
\begin{tabular}{llrrr}
\toprule
 & skip\_type & proj & skip & var \\
 & skip\_num &  &  &  \\
\midrule
\multirow[t]{5}{*}{Evaluation Jaccard Index} & 1 & 0.412 & \textbf{0.426} & 0.410 \\
 & 2 & \textbf{0.462} & 0.452 & 0.429 \\
 & 3 & 0.474 & \textbf{0.493} & 0.463 \\
 & 4 & 0.456 & \textbf{0.494} & 0.434 \\
 & 5 & 0.440 & \textbf{0.490} & 0.433 \\
\cline{1-5}
\multirow[t]{5}{*}{Parameters (x$1e^6$)} & 1 & 67.160 & \textbf{29.410} & 104.911 \\
 & 2 & 78.958 & \textbf{31.769} & 126.147 \\
 & 3 & 81.907 & \textbf{32.359} & 131.456 \\
 & 4 & 82.645 & \textbf{32.506} & 132.784 \\
 & 5 & 82.700 & \textbf{32.525} & 132.876 \\
\cline{1-5}
\multirow[t]{5}{*}{Total MAC (x$1e^9$)} & 1 & 2.791 & \textbf{2.187} & 3.395 \\
 & 2 & 3.546 & \textbf{2.338} & 4.754 \\
 & 3 & 4.301 & \textbf{2.489} & 6.113 \\
 & 4 & 5.056 & \textbf{2.640} & 7.473 \\
 & 5 & 5.283 & \textbf{2.715} & 7.851 \\
\cline{1-5}
\bottomrule
\end{tabular}
\end{table}


Again an ANOVA is fitted to determine the significant effects, the results can be seen in Table~\ref{tab:skip_importance_anova_all}. 
Then, an OLS is fitted to show the effect size of all significant factors, the results of which can be seen in Table~\ref{tab:skip_importance_ols_effects}. From this limited sample of runs, the type does not seem to be significant. Neither does the number of skip connections prove to be a significant effect based on the OLS. However, they are both close to a significance value of $0.05$, thus a bigger sample might show a small (likely positive) effect. Moreover, the increase in skip connections also increases the inference complexity. Hence, it might not be worth the reduced inference speed.

\begin{table}[ht]
\caption{Anova results estimating the influence of each parameter.\\Where: \\\hphantom{tabb}`weights' are the pretrained weights (or lack thereof) used.\\\hphantom{tabb}`architecture' is the model architecture used.\\\hphantom{tabb}'frozen' indicates whether the encoder was frozen\\\hphantom{tabb}$A$:$B$ is the interaction effect between $A$ and $B$}
\label{tab:comparison_baselines_anova_all}
\begin{tabular}{lrrrrr}
\toprule
 & df & sum\_sq & mean\_sq & F & PR(>F) \\
\midrule
frozen & 1.00 & 0.00 & 0.00 & 5.29 & 0.08 \\
weights & 3.00 & 0.25 & 0.08 & 175.48 & \textbf{0.00} \\
architecture & 2.00 & 0.01 & 0.01 & 12.69 & \textbf{0.02} \\
frozen:weights & 3.00 & 0.00 & 0.00 & 1.53 & 0.34 \\
frozen:architecture & 2.00 & 0.00 & 0.00 & 2.16 & 0.23 \\
weights:architecture & 6.00 & 0.00 & 0.00 & 0.64 & 0.70 \\
Residual & 4.00 & 0.00 & 0.00 & n.a. & n.a. \\
\bottomrule
\end{tabular}
\end{table}


\begin{table}[ht]
\caption{Coefficients of the OLS showing the influence of the hyperparameters on the Evaluation Jaccard Index.\\Where:\\\hphantom{tabb}Coef. the effectsize.\\\hphantom{tabb}P> |t| the p-value. Bolded if significant ($\alpha\le0.05$).}
\label{tab:comparison_baselines_ols_effects}
\begin{tabular}{lrrrrrr}
\toprule
 & Coef. & Std.Err. & t & P>|t| & [0.025 & 0.975] \\
\midrule
Intercept & 0.22 & 0.02 & 10.54 & \textbf{0.00} & 0.17 & 0.26 \\
weights[T.imagenet] & 0.21 & 0.02 & 9.59 & \textbf{0.00} & 0.17 & 0.26 \\
weights[T.vae-b1] & -0.03 & 0.02 & -1.33 & 0.20 & -0.08 & 0.02 \\
weights[T.vae-b100] & -0.03 & 0.02 & -1.54 & 0.14 & -0.08 & 0.01 \\
architecture[T.fpn] & 0.06 & 0.02 & 3.49 & \textbf{0.00} & 0.02 & 0.09 \\
architecture[T.unet] & 0.02 & 0.02 & 1.48 & 0.16 & -0.01 & 0.06 \\
\bottomrule
\end{tabular}
\end{table}



\subsection{Backbone}
We first analyse the influence of the backbone on the performance of the VAE task by comparing the following backbones: MobileNetV2~\cite{sandler2019mobilenetv2invertedresidualslinear}, EfficientNet~\cite{tan2020efficientnetrethinkingmodelscaling}, and two variants of the ResNet~\cite{he2015deep} architecture. The results can be seen in Table~\ref{tab:vae-backbones-results}. Some examples of the reconstruction capabilities can be seen in Figure~\ref{fig:vae-backbones}. MobileNetV2 is the smallest network, as can be seen from the parameter count and Multiply-Accumulate (MAC) values. Overall ResNet18 achieves the best L1Loss, meaning it is best at reconstruction. Although the difference is small. Visually, all networks perform very similar.

\begin{figure}[!ht]
    \centering
    \subfloat[Original, EfficientNet]{\includegraphics[width=0.45\linewidth]{figures/vae-backbones/reconstruction/samples/efficientnet_b2/14.png}}
    \subfloat[Original, MobileNetV2]{\includegraphics[width=0.45\linewidth]{figures/vae-backbones/reconstruction/samples/mobilenetv2_100/14.png}} \quad
    \subfloat[Original, ResNet18]{\includegraphics[width=0.45\linewidth]{figures/vae-backbones/reconstruction/samples/resnet18/14.png}}
    \subfloat[Original, ResNet50]{\includegraphics[width=0.45\linewidth]{figures/vae-backbones/reconstruction/samples/resnet50/14.png}}
    \caption{Example reconstruction for the various backbones.}
    \label{fig:vae-backbones}
\end{figure}

\begin{table}[ht]
\centering
\caption{VAES results of the various backbones.}
\label{tab:vaes-backbones-results}
\begin{tabular}{lrrr}
\toprule
 & Parameters ($1e^6$) & MAC ($1e^9$) & Jaccard Index \\
Backbone &  &  &  \\
\midrule
MobileNetV2 & \textbf{6.05} & \textbf{0.77} & 0.42 \\
ResNet18 & 20.67 & 1.96 & 0.42 \\
ResNet50 & 132.87 & 7.80 & \textbf{0.46} \\
\bottomrule
\end{tabular}
\end{table}


We redo the analysis for the VAES task with the same backbones. Examples of the segmentation masks can be seen in Figure~\ref{fig:vaes-backbones} The results can be seen in Table~\ref{tab:vaes-backbones-results}. For semantic segmentation, the smaller architectures lack behind ResNet50. However, depending on the computational limitations, it might be worth to use the smaller networks as the difference is not huge.

\begin{figure}[h]
    \foreach \i in {14,11} {
            \centering
            \subfloat[Image]{\includegraphics[width=0.16\textwidth]{figures/vae-backbones/semantic/samples/resnet50/\i.png}}
            \subfloat[Ground Truth]{\includegraphics[width=0.16\textwidth]{figures/vae-backbones/semantic/samples/resnet50/gt_\i.png}}
            \subfloat[ResNet50]{\includegraphics[width=0.16\textwidth]{figures/vae-backbones/semantic/samples/resnet50/pr_\i.png}}
            \subfloat[ResNet18]{\includegraphics[width=0.16\textwidth]{figures/vae-backbones/semantic/samples/resnet18/pr_\i.png}}
            \subfloat[MobileNetV2]{\includegraphics[width=0.16\textwidth]{figures/vae-backbones/semantic/samples/mobilenetv2_100/pr_\i.png}}
            \subfloat[!!TODO!!]{\includegraphics[width=0.16\textwidth]{figures/vae-backbones/semantic/samples/mobilenetv2_100/pr_\i.png}}
            \\
        }
    \caption{Samples of the validation dataset, with the ground truth and the prediction by VAES with different backbones.}\label{fig:vaes-backbones}
\end{figure}

\begin{table}[ht]
\centering
\caption{VAES results of the various backbones.}
\label{tab:vaes-backbones-results}
\begin{tabular}{lrrr}
\toprule
 & Parameters ($1e^6$) & MAC ($1e^9$) & Jaccard Index \\
Backbone &  &  &  \\
\midrule
MobileNetV2 & \textbf{6.05} & \textbf{0.77} & 0.42 \\
ResNet18 & 20.67 & 1.96 & 0.42 \\
ResNet50 & 132.87 & 7.80 & \textbf{0.46} \\
\bottomrule
\end{tabular}
\end{table}


\subsection{$\beta\text{-value}$}
We also analyse the effect of the $\beta$ factor on the reconstruction quality of the VAE. This is done by training models with varying values for $\beta$. The results can be seen in Table~\ref{tab:beta-vae-loss-values}. Some examples of the reconstruction capabilities can be seen in Figure~\ref{fig:beta-vae-recon-examples}. More samples can be seen in Appendix~\ref{appendix:recon_samples}. Based on the images and the reconstruction error, it can be concluded that the higher values for $\beta$ the reconstruction capabilities of the network are reduced. This is to be expected as the model is forced to use a 

\begin{table}[!ht]
    \centering
    \caption{Loss values resulting from training a Beta-VAE for various $\beta$ values}
    \label{tab:beta-vae-loss-values}
    \begin{tabular}{ccc}
        \hline
        $\beta$ & KL-Divergence & Reconstruction Error ($1e^5$) \\
        \hline
        0.01    & 80200         & 1.9                           \\
        0.1     & 31250         & 1.5                           \\
        1       & 8190          & 2.4                           \\
        10      & 2089          & 2.2                           \\
        100     & 498           & 2.9                           \\
        \hline
    \end{tabular}
\end{table}

\begin{figure}[!ht]
    \centering
    \caption{Example reconstructions for $\beta$-vae.}
    \label{fig:beta-vae-recon-examples}
    \subfloat[Original, $\beta$ = 0.01]{\includegraphics[width=0.45\linewidth]{figures/beta-vae/b0.01-0.png}}
    \subfloat[Original, $\beta$ = 0.1]{\includegraphics[width=0.45\linewidth]{figures/beta-vae/b0.1-0.png}} \quad
    \subfloat[Original, $\beta$ = 1]{\includegraphics[width=0.45\linewidth]{figures/beta-vae/b1-0.png}}
    \subfloat[Original, $\beta$ = 10]{\includegraphics[width=0.45\linewidth]{figures/beta-vae/b10-0.png}}
\end{figure}
